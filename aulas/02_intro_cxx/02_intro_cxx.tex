
\documentclass[xcolor={usenames,dvipsnames},10pt,compress,aspectratio=169]{beamer}

\usepackage[utf8]{inputenc}
\usepackage[brazilian]{babel}
\usepackage{verbatim}
\usepackage{graphicx}
\usepackage{xspace}
\usepackage{amsthm}
\usepackage{url}
\usepackage{array}
\usepackage{hyperref}
\usepackage{times,mathptmx}
\usepackage{pdfpages}
\usepackage{mdframed}
\usepackage{tikz}
\usepackage{alltt}
%\usepackage[usenames,dvipsnames]{xcolor}
%\usepackage[usenames,dvipsnames]{color}
%\usepackage{color}

\usetikzlibrary{arrows,shapes}

\usetheme{Madrid}
%\usetheme{Boadilla}
%\usetheme{Darmstadt}
%\usetheme{Frankfurt}
%\usetheme{CambridgeUS}
%\usetheme{AnnArbor}
%\usecolortheme{beaver}
%\usecolortheme{seahorse}
%\usecolortheme{seagull}

\setbeamercovered{transparent}

\setbeamertemplate{footline}[frame number]
%\setbeamertemplate{navigation symbols}{}
%\setbeamersize{text margin left=1em,text margin right=1em}

\newcommand{\titulo}{Introdução C++}
\newcommand{\disciplina}{ELC1067 - Laboratório de Programação II}
\newcommand{\nome}{João Vicente Ferreira Lima (UFSM)}

\lecture[1]{\aula}{aula01}
\def\lecturename{\aula}

\newcommand{\Red}[1]{{\color{red}#1}}
\newcommand{\red}[1]{{\color{red}#1}}
\newcommand{\Blue}[1]{{\color{blue}#1}}
\newcommand{\blue}[1]{{\color{blue}#1}}

\newcommand{\PBS}[1]{\let\temp=\\#1\let\\=\temp}
\newcommand{\RRCOL}{\PBS\raggedright\hspace{0pt}}

\newcommand{\p}[1]{\texttt{#1}}
\newenvironment{code}{%
  \begin{alltt}%
  }{%
  \end{alltt}%
}

\makeatletter
%\setbeamertemplate{headline}{}
% {%
%   \leavevmode%
%   \@tempdimb=2.4375ex%
%   \ifnum\beamer@subsectionmax<\beamer@sectionmax%
%     \multiply\@tempdimb by 4%
%   \else%
%     \multiply\@tempdimb by\beamer@subsectionmax%
%   \fi%
%   \ifdim\@tempdimb>0pt%
%     \advance\@tempdimb by 1.125ex%
%     \begin{beamercolorbox}[wd=.5\paperwidth,ht=\@tempdimb]{section in head/foot}%
%       \vbox to\@tempdimb{\vfil\insertsectionnavigation{.5\paperwidth}\vfil}%
%     \end{beamercolorbox}%
%     \begin{beamercolorbox}[wd=.45\paperwidth,ht=\@tempdimb]{subsection in head/foot}%
%       \vbox
%       to\@tempdimb{\vfil\insertsubsectionnavigation{.45\paperwidth}\vfil}%
%     \end{beamercolorbox}%
%     \begin{beamercolorbox}[wd=.05\paperwidth,ht=\@tempdimb]{subsection in head/foot}%
%       \vbox
%       to\@tempdimb{\vfil\hfil\insertframenumber\vfil\vfil}%
%     \end{beamercolorbox}%
%   \fi%
% }

\def\dohead{\beamer@headcounter=4\relax\beamer@headcounter=1\loop\ifnum\beamer@headcounter<\beamer@totalheads%
  \advance\beamer@headcounter by1\relax%
  \csname @@head\the\beamer@headcounter\endcsname\repeat}

\makeatother

\title[\titulo]{\titulo}

\subtitle{\disciplina}

\author[João V. F. Lima]{\nome}

\institute[UFSM]{Departamento de Linguagens e Sistemas de Computação \\ Universidade Federal de Santa Maria \\ \url{jvlima@inf.ufsm.br} \\ \url{http://www.inf.ufsm.br/~jvlima}}
\date{2020/2}

\graphicspath{{.}{figs/}}

\newtheorem{mydef}{Definição}[section]
%\newtheorem{myteo}{Teorema}[section]
%------------------------------------------------------------------------------
%\newcommand{\xkaapi}{XKaapi\xspace}
%------------------------------------------------------------------------------
% Typesetting Listings
\usepackage{listings}
\lstset{
  language=C++,
  %basicstyle=\scriptsize\ttfamily,
  basicstyle=\normalsize\ttfamily,
  %basicstyle=\small\ttfamily,
  %basicstyle=\footnotesize\ttfamily,
  aboveskip=0pt,
  belowskip=0pt,
  mathescape=false,
  columns=flexible,
%  numbers=none,
  numbers=left,
%  showtabs=true,
%  showspaces=true,
  breaklines=true
}
%------------------------------------------------------------------------------
\lstset{commentstyle=\color{blue}}
%\lstset{stringstyle=\ttfamily}
%\lstset{ classoffset=1, 
%            morekeywords={kaapi,omp,task,data,alloca, declare, reduction, identity, parallel,sync,taskwait,cilk,spawn,tbb,css,cilk\_spawn,cilk\_sync,cilk\_for,offload},
%            keywordstyle=\color{Red}\bfseries
%           }
%\lstset{ classoffset=2, 
%            morekeywords={value,read,write,readwrite,reduction,untied,firstprivate,TaskBodyCPU,TaskBodyGPU,ka,Signature,RW,CW,range2d\_r,range2d\_rw,range2d,Spawn,Fork,Shared\_w,Shared\_r,Shared,a1,target,device,copyin,copyout,input,implements,copy\_deps,RPWP,range2d\_rpwp,rangeindex,Memory,Register,SetStaticSched,Sync,Unregister,Community,System,join\_community,SpawnMain,leave,initialize,terminate,logfile,array,SetArch,ArchHost,ArchCUDA,W,R,gpuStream,pointer\_w,pointer\_r,pointer\_cw,pointer},
%            keywordstyle=\color{Blue}\bfseries
%           }
%\lstset{ classoffset=3, 
%            morekeywords={storage,ld},
%            keywordstyle=\bfseries
%           }
%\lstset{ classoffset=4, 
%            morekeywords={in,out,inout,cout,concurrent},
%            keywordstyle=\color{Red}\bfseries
%           }
%           
\lstset{classoffset=0, showstringspaces=false}
%------------------------------------------------------------------------------
\mdfsetup{
  backgroundcolor=gray!10,
%  roundcorner=10pt,
}
%------------------------------------------------------------------------------
\newcommand{\restorefootline}{\setbeamertemplate{navigation symbols}{}}
%\newcommand{\setfootline}[1]{\setbeamertemplate{navigation symbols}{\textcolor{black}{\textbf{#1}}}}
\newcommand{\includeslides}[4]{%
  %\setfootline{#1}%
  {
    \setbeamercolor{background canvas}{bg=}
    \includepdf[pages={#2},%
    pagecommand={},
%    pagecommand={\begin{frame}[default]{}\end{frame}},
    #4,%
    turn=false,noautoscale=false,column=false,columnstrict=false,openright=false,frame=false]{#3}%
  }
  %\restorefootline%
}
%------------------------------------------------------------------------------
\begin{document}

\begin{frame}
%  \titlepage
  \maketitle
  \mode<presentation>
  {
    \begin{columns}
      \begin{column}{0.5\textwidth}
      \raggedleft
	\includegraphics[width=2cm]{logo_ufsm}
      \end{column}
      \begin{column}{0.5\textwidth}
	\includegraphics[width=2cm]{logo_inf}
      \end{column}
    \end{columns}
  }
\end{frame}

\begin{frame}
    \frametitle{Outline}
    \tableofcontents[hideallsubsections]
%    \tableofcontents
\end{frame}

\AtBeginSection{
  \begin{frame}
    \frametitle{Outline}
    \tableofcontents[currentsection,hideothersubsections]
  \end{frame}
}

%%%%%%%%%%%%%%%%%%%%%%%%%%%%%%%%%%%%%%%%%%%%%%%%%%%%%%%%%%%%%%%%%%%%%%%%%%%%%%%
\section{Introdução}
%%%%%%%%%%%%%%%%%%%%%%%%%%%%%%%%%%%%%%%%%%%%%%%%%%%%%%%%%%%%%%%%%%%%%%%%%%%%%%%

%------------------------------------------------------------------------------
%\begin{frame}
%  \frametitle{Entrada e saída}
%  \begin{itemize}
%  \item 
%  \end{itemize}
%\end{frame}
%------------------------------------------------------------------------------
%------------------------------------------------------------------------------
\begin{frame}[fragile]
  \frametitle{Olá mundo}
  \vspace{-2mm}
  \begin{block}{ola.cpp}
\begin{lstlisting}
#include <iostream>

int main(void)
{
    std::string mensagem{"Ola mundo!"}
    // isso eh um comentario
    std::cout << "Saida: " << mensagem << std::endl;
    return 0;
}
\end{lstlisting}
\end{block}
%
\begin{exampleblock}{Comentários}
\begin{itemize}
\item \textbf{iostream} para entrada e saída básica.
\item \textbf{std::string} é uma estrutura (``classe'') C++ para string.
\item \textbf{std::cout} é a saída padrão.
\item \textbf{std::endl} é uma nova linha (\texttt{\char`\\n}).
\end{itemize}
\end{exampleblock}
%
\end{frame}
%------------------------------------------------------------------------------
\begin{frame}[fragile]
  \frametitle{Compilação}
\begin{exampleblock}{Linha de comando}
\begin{lstlisting}
$ g++ -std=c++11 -O2 -Wall -g  -o ola ola.cpp
\end{lstlisting}
\end{exampleblock}
%
\begin{alertblock}{Dica}
Semelhante ao C, pode-se usar o \textbf{GCC} (\texttt{g++}) e \textbf{Clang}
(\texttt{clang++}).
Clang mostra erros de compilação de forma mais amigável, além de ser mais 
eficiente que o GCC.
\end{alertblock}
\end{frame}
%------------------------------------------------------------------------------
\begin{frame}[fragile]
  \frametitle{Linha de comando}
  \begin{block}{params.cpp}
\begin{lstlisting}
#include <iostream>

// argc eh o numero de parametros passados
// argv eh um vetor de strings com os valores
int main(int argc, char **argv)
{
    for(auto i= 0; i < argc; i++){
        std::cout << "Param " << i
            << " valor -> " << argv[i] << std::endl;
    }
    return 0;
}
\end{lstlisting}
  \end{block}
\end{frame}
%------------------------------------------------------------------------------
%%%%%%%%%%%%%%%%%%%%%%%%%%%%%%%%%%%%%%%%%%%%%%%%%%%%%%%%%%%%%%%%%%%%%%%%%%%%%%%
\section{Variáveis}
%%%%%%%%%%%%%%%%%%%%%%%%%%%%%%%%%%%%%%%%%%%%%%%%%%%%%%%%%%%%%%%%%%%%%%%%%%%%%%%
%------------------------------------------------------------------------------
\begin{frame}[fragile]
  \frametitle{\texttt{auto}}
Pode-se usar \textbf{auto} quando o tipo é deduzido pelo compilador.
  \begin{block}{Exemplo}
\begin{lstlisting}
auto x = 1;          // inteiro
auto y = 2.0;        // double
auto teste = true;   // booleano

// i abaixo eh um inteiro
for(auto i= 0; i < 10; i++)
    std::cout << "Valor: " << i << std::endl;
\end{lstlisting}
  \end{block}
\end{frame}
%------------------------------------------------------------------------------
\begin{frame}[fragile]
  \frametitle{Inicialização padrão}
É uma forma de padronizar a inicialização de variáveis em C++ usando \verb+{}+.
  \begin{block}{Exemplos}
\begin{lstlisting}
double x {1.0};                 // declara um double
int    a[] {1, 2, 3, 4};        // vetor com 4 elementos sem = 
int    b[] = {1, 2, 3, 4};      // mesma coisa
std::string nome {"Meu nome"} ; // uma string
\end{lstlisting}
  \end{block}
  %
\begin{alertblock}{Atenção}
Não funciona com \textbf{auto}.
\begin{lstlisting}

auto x {1.0}; // double ou float ?
\end{lstlisting}
\end{alertblock}
\end{frame}
%------------------------------------------------------------------------------
\begin{frame}[fragile]
  \frametitle{Casting (conversão)}
C++ apresenta quatro tipos de conversão:
\begin{itemize}
\item \textbf{static\_cast} - tipos relacionados: \texttt{int} para \texttt{char},
ou \texttt{double*} para \texttt{int*}.
\item \textbf{reinterpret\_cast} - tipos não relacionados (inteiro para ponteiro, etc).
\item \textbf{const\_cast} - \texttt{const} ou \texttt{volatile}.
\item \textbf{dynamic\_cast} (\emph{não usado aqui}).
\end{itemize}
  \begin{block}{Exemplo}
\begin{lstlisting}
int num = 97;                        // inteiro
char letra = static_cast<char>(num); // agora letra A

char *dados = new char[100];          // 100 chars alocados
int* vetor = reinterpret_cast<int*>(dados); // mudei agora para int
\end{lstlisting}
  \end{block}
\end{frame}
%------------------------------------------------------------------------------
\begin{frame}[fragile]
  \frametitle{Passagem por referência}
Passagem por referência possibilita passar variáveis por 
\textbf{referência} (\verb+&+) ao invés de valor ou ponteiro.
  \begin{block}{Exemplo}
\begin{lstlisting}
void f(int val, int& ref)
{
    val++;   // incrementa a copia local de val
    ref++;   // incrementa realmente a variavel
}
\end{lstlisting}
  \end{block}
%
\end{frame}
%------------------------------------------------------------------------------
\begin{frame}[fragile]
  \frametitle{Passagem por referência}
\begin{alertblock}{Importante}
Evite usar passagem por referência porque deixa o programa
mais difícil de entender.
Use apenas quando queremos evitar uma cópia e não vamos alterar a variável (\verb+const+), 
como por exemplo um vetor ou uma string:
\begin{lstlisting}

void imprimir(const std::string& texto) 
{
    std::cout << texto << std::endl; // nao altera variavel
}
\end{lstlisting}
\end{alertblock}
\end{frame}
%------------------------------------------------------------------------------
\begin{frame}[fragile]
  \frametitle{Estrutura de dados}
  \begin{block}{struct Ponto}
\begin{lstlisting}
struct Ponto {
    float x;  // variaveis
    float y;

    // zera o ponto 
    void zera(void) {
        x = 0.0f;
        y = 0.0f;
    }
    // distancia deste ponto (x, y) ate p1
    float distancia(Ponto& p1) const {
        return std::sqrt( std::pow((x-p1.x), 2) + std::pow((y-p1.y), 2) );
    }
};
\end{lstlisting}
  \end{block}
\end{frame}
%------------------------------------------------------------------------------
\begin{frame}[fragile]
  \frametitle{Estrutura de dados}
  \begin{block}{struct Ponto}
\begin{lstlisting}
int main(void)
{
    Ponto p1 {1.0, 1.0};
    Ponto p2;
    p2.zera();
    p2.x = 19.0;
    p2.y = 20.0;

    auto distancia = p1.distancia(p2);
    std::cout << "Distancia: " << distancia << std::endl;
}
\end{lstlisting}
  \end{block}
\end{frame}
%------------------------------------------------------------------------------
%%%%%%%%%%%%%%%%%%%%%%%%%%%%%%%%%%%%%%%%%%%%%%%%%%%%%%%%%%%%%%%%%%%%%%%%%%%%%%%
\section{Entrada e saída}
%%%%%%%%%%%%%%%%%%%%%%%%%%%%%%%%%%%%%%%%%%%%%%%%%%%%%%%%%%%%%%%%%%%%%%%%%%%%%%%
%------------------------------------------------------------------------------
\begin{frame}[fragile]
  \frametitle{Entrada e saída}
As operações são efetuadas por \textbf{streaming} ou fluxo onde dados:
\begin{itemize}
\item \textbf{std::ifstream} para leitura (operador \verb+>>+).
\item \textbf{std::ofstream} para escrita (operador \verb+<<+).
\end{itemize}
  \begin{block}{Exemplo}
\begin{lstlisting}
#include <iostream>
#include <fstream>

int main(void)
{
    int n1, n2;
    std::ifstream entrada {"entrada.txt"};
    std::ofstream saida {"saida.txt"};
    entrada >> n1 >> n2;
    saida << n1 << " " << n2 << std::endl;
    return 0;
}
\end{lstlisting}
  \end{block}
\end{frame}
%------------------------------------------------------------------------------
\begin{frame}[fragile]
  \frametitle{Entrada e saída}
  \begin{block}{EOF - \emph{end-of-file}}
\begin{lstlisting}
#include <iostream>
#include <fstream>
int main(void) {
    int n;
    std::ifstream entrada {"numeros.txt"};
    std::ofstream saida {"saida.txt"};
    if(entrada.is_open() == false)
        throw std::runtime_error{"ERRO arquivo!"};

    while(entrada.eof() == false){
        entrada >> n;
        saida << n << std::endl;
    }
    entrada.close();
    saida.close();
    return 0;
}
\end{lstlisting}
  \end{block}
\end{frame}
%------------------------------------------------------------------------------
\begin{frame}[fragile]
  \frametitle{Entrada e saída}
  \begin{block}{Ler linha em C++ (continua)}
\begin{lstlisting}
#include <iostream>
#include <fstream>
#include <vector>

struct Aluno {
    int matricula;
    std::string nome;
};
\end{lstlisting}
  \end{block}
\end{frame}
%------------------------------------------------------------------------------
\begin{frame}[fragile]
  \frametitle{Entrada e saída}
  \begin{block}{Ler linha em C++}
\begin{lstlisting}
int main(void)
{
    int matricula;
    std::string nome;
    std::vector<Aluno> alunos;             // vetor de alunos
    std::ifstream entrada {"alunos.txt"};
    while( entrada >> matricula ) { // le matricula 
        std::getline(entrada, nome); // le resto da linha
        alunos.push_back( Aluno{matricula, nome} );
    }

    for(Aluno& v: alunos)
        std::cout << v.matricula << " -> " << v.nome << std::endl;
    return 0;
}
\end{lstlisting}
  \end{block}
\end{frame}
%------------------------------------------------------------------------------
\begin{frame}[fragile]
  \frametitle{Namespaces}
Namespaces em C++ criam \textbf{escopos nomeados}  e aumenta a modularidade do código.
A \verb+std+ é o namespace padrão do C++.
  \begin{block}{Exemplo}
\begin{lstlisting}
namespace Uteis {
    void foo(void) {
        std::cout << "Funcao foo aqui" << std::endl;
    }
}

// usando a funcao assim
Uteis::foo();
\end{lstlisting}
  \end{block}
\end{frame}
%------------------------------------------------------------------------------

\end{document}
